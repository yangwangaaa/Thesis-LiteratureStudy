\chapter{Reinforcement Learning Preliminaries}
This chapter is dedicated to present a concise theory of reinforcement learning. The first section will show how a certain goal can be formalized as a reward maximization -- one of the ideas which serves as a basic foundation of \ac{RL}. Section \ref{sec:mdp} explains the basics of \ac{MDP}, a general framework used in \ac{RL} problem. Subsequently, an intuition of value and policy iteration will be developed in section \ref{sec:value_iter}. The fourth section will present the extension of \ac{RL} for continuous space. Finally, section \ref{sec:actor} will discuss the actor-critic structure which is a natural representation for control system problem.

\section{The Principle of Maximizing Cumulative Reward}
The nature of \ac{RL} is inspired by the way living organisms learn to reach their desired goals by first acting on the environment, observe the changes that occur, and reward their action accordingly. if tThe idea of \ac{RL} is to 

\section{Markov Decision Process} \label{sec:mdp}
\ac{MDP} is defined as a tuple which satisfies Markov property. The detailed explanation of this property can be found on \cite{sutton1998reinforcement} section 3.5 but the main idea is that to determine the probability of a state at certain time, it is sufficient to know only the state of previous time instant. This probability is mathematically denoted in Equation \eqref{eq:markov}.

\begin{equation}
	\text{Pr}\{x_{t+1} = x', r_{t+1} = r| x_t, u_t \}
	\label{eq:markov}
\end{equation}
where $x$ denotes state, $u$ denotes action, and $r$ denotes immediate reward obtained upon applying the input on the corresponding state.

\section{Value and Policy Iteration} \label{sec:value_iter}

\section{Reinforcement Learning for Continuous Space}
\subsection{Function Approximation}


\section{Actor-Critic Structure} \label{sec:actor}
