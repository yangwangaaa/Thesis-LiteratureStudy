%
% Abstract (does not appear in the Table of Contents)
\chapter*{Abstract}%

Reference or trajectory tracking is one of the requirements in order to carry out a complex robotic task. Capability to perform a precise tracking with minimum possible error is crucial for the robots that are to be deployed at manufacturing industries such as semiconductor, automotive and recently, an emerging application of \ac{3D} printing.

The approach used in the past has been to design model based controllers which involve feedback and feedforward control or more recently, a predictive control. The drawback of such scheme, however, lies on the requirements of system model as a slight model mismatch could lead to poor tracking performance. For a repetitive control error, researchers have designed the so called \ac{ILC}. In this literature study, a new method to optimize the tracking performance of nominal controller using \ac{RL} is proposed. 

Throughout the literature study, the existing work for \acs{RL}-based tracking control clusters into 3 approaches: \acs{RL} for optimal tracking (Kiumarsi et al. \cite{Kiumarsi6760476}), \acs{RL}-based dynamic tuning (Brujeni et al. \cite{Brujeni5669655}), and nonlinear compensator via \acs{RL} (Bayiz et al. \cite{Efe2014}). The advantages, limitations and practical challenges of the 3 approaches are discussed. These criterion serves as a basis to select one method which will be developed and implemented during the thesis. Furthermore, the testbed for the thesis which is a UR5 \acs{3D} printing robot is also presented. Finally, the literature study concludes with the research plan and discussion.