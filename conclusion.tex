\chapter{Conclusion}
Despite having used for various applications for decades, the application of \ac {RL} in control still has much space to explore. In this thesis, a subset of that space, namely reference tracking problem is addressed. The search for existing works results in 3 methods which must be analyzed carefully in order to obtain a strong motivation to finally choose one of them. The \ac {RL}-based optimal tracking is mathematically more convincing than the other two. Not only that it will surely converge, a stability can also be guaranteeed. In the other hand, the dynamic tuning methods have problems with performance and feasibility. The direct tuning is intuitive and relatively easy to implement, but it will perform less since it depends on a feedback controller which will always be late in compensating error. The \ac{PI$^2$} method has successful track-record for variable impedance control task, but rather unconvincing for tracking problem. Finally, the additive tracking problem is chosen since it strongest represents the nature of \ac {RL} which does not depend on the system model. This method enables an additional degree of freedom to optimize the controller, independent from the nominal controller itself.